\chapter{I Kit Precampionati}


\xentrystretch{-0.05}
\tablehead{%
%\hline
\multicolumn{1}{l}{\bf{Macchina}} &
\multicolumn{1}{l}{\bf{Anno}} &
\multicolumn{1}{l}{\bf{Informazioni}}
\\
}
%\tabletail{ \hline }
\tablelasttail{ \hline }
%\tablelasttail{ \multicolumn{3}{l}{} \\ \hline }
\begin{xtabular}{p{2.02cm}|l|p{7.8cm}}

\hline
SP0256-AL2 \linebreak
General Instruments 
& 1981 & 
\includegraphics[width=4.7cm]{sp0256al2}

L’integrato SP0256-AL2 Speech Processor contiene un filtro digitale programmabile che può modellare un tratto vocale. La ROM da 16k conserva dati e istruzioni. L’output a modulazione d'impulso può produrre del parlato con un range di frequenza di 5kHz e un range dinamico di 42 dB. \\
\hline
TR-606 \linebreak Roland & 1981 & 
\includegraphics[width=5.8cm]{tr606}

La Drumatix TR-606 della Roland è una drum machine analogica programmabile. Fu progettata per suonare assieme al TB-303 Bassline. La TR-6060 ha un suono molto originale ed è tuttora molto popolare. \\
\hline
TR-707 \linebreak Roland & 1984 & 
\includegraphics[width=5.8cm]{tr707}

La Roland TR-707 ha le stesse funzioni della TR-909 con tutti i suoni in PCM. A partire da questo modello, la Roland cominciò ad implementare uno schermo LCD per mostrare la matrice del ritmo e il tempo. \\
\hline
TR-727 \linebreak Roland & 1985 & 
\includegraphics[width=5.8cm]{tr727}

La Roland TR-727 è identica alla TR-707, eccetto per il fatto che i suoi suoni sono percussioni etniche/latine. È stata pensata per accompagnare una sezione ritmica, piuttosto che per essere utilizzata da sola. \\
\hline
TR-808 \linebreak Roland & 1980 & 
\includegraphics[width=6.0cm]{tr808}

La Roland TR-808 ha giocato un ruolo fondamentale per i movimenti Hip Hop ed Electro degli anni ’80. È ancora molto popolare grazie ai suoi inconfondibili suoni originali. \\
\hline
TR-909 \linebreak Roland & 1983 & 
\includegraphics[width=5cm]{tr909}

La Roland TR-909 è una delle drum machine più popolari di sempre. Utilizza campioni PCM per i cimbali e gli hi-hat, ma tutti gli altri strumenti sono ancora riprodotti da circuiteria analogica. I suoi suoni sono molto utili per fare musica Techno e House. \\
\hline
CR-78 \linebreak Roland & 1978 & 
\includegraphics[width=5cm]{cr78}

La Roland CR-78 fu forse la più lussuosa rythm machine mai creata. Il güiro e il tamburello hanno ancora suoni unici al giorno d’oggi, mentre il basso, lo snare ed i bonghi sono molto morbidi e ricchi di suono. \\
\hline
CR-8000 \linebreak Roland & 1981 & 
\includegraphics[width=5cm]{cr8000}

La Roland CR-8000 fu introdotta dopo la TR-808 –- ha lo stesso motore analogico. Lo hi-hat suona più realistico che in altre rhythm machine più vecchie, ma il battito di mani ricorda quasi uno snare elettrico. \\
\hline
DR-55 \linebreak Boss & 1979 & 
\includegraphics[width=5.4cm]{dr55}

Le drum machine della serie Dr. Rhythm di Boss fu progettata specificamente per i chitarristi che necessitavano di un batterista portatile. La DR-55 è una semplice drum machine analogica caratterizzata da un suono molto grezzo e diretto. \\
\hline
DR-110 \linebreak Boss & 1983 & 
\includegraphics[width=5.4cm]{dr110}

La DR-110, successiva alla DR-55, dispone di un suono analogico, ma si programma digitalmente mediante una matrice ritmica digitale su schermo LCD. Molto probabilmente offre il migliore battito di mani analogico di sempre. \\
\hline
LinnDrum & 1982 & 
\includegraphics[width=6.1cm]{linndrum}

La LinnDrum originariamente fu messa in vendita per 3000\$ e ne furono prodotte circa 5000 unità. Essa diede vita alle tracce ritmiche di moltissime hit degli anni '80. \\
\hline
Rhythm Ace & 1967 & 
\includegraphics[width=5.2cm]{rhyace}

La Ace Tone fu la prima compagnia a produrre rhythm box elettroniche in Giappone. Nel Regno Unito, Bentley Pianos (che metteva etichette su tutti i suoi prodotti) si occupò della distribuzione della Ace Tone, che dunque divenne famosa anche col nome di Bentley Rhythm Ace. \\
\hline
Tom \linebreak
Sequential \linebreak
Circuits & 1984 & 
\includegraphics[width=6.7cm]{tom}

I suoi suoni sono un po' sporchi e grezzi, specialmente se confrontati con la sorella maggiore Drumtraks, ma è ciò che conferisce al Tom il suo carattere. Gli snare suonano come nient'altro su questo pianeta - sono elettrici! \\
\hline
Acieed House & 1990's & 
\includegraphics[width=3.5cm]{smiley_gray}

Questo insieme di campioni vocali derivano da un svariati pezzi Acid House famosi. \textit{Can you dig it?} \\
\hline
Ghetto Bass & 1990's & 
\includegraphics[width=5.0cm]{booty} 

Un mucchio di campioni derivanti da classici della ghetto house di Detroit/Chicago.\\
\hline
Animals \linebreak di Bud Melvin & 2004 & 
\includegraphics[width=6.7cm]{kit-farm} 

Il vincitore della Animal Sample Compo del 2004. Una pregevole selezione di animali domestici. \\
\end{xtabular}

