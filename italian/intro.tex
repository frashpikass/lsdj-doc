\chapter{Introduzione}
\section{Ciao!}
Prima di tutto, grazie di aver deciso di provare Little Sound Dj!

Un sacco di sforzi sono stati fatti per rendere questo programma il più veloce e potente possibile.
Se non avete esperienza con l’interfaccia degli editor musicali denominati “tracker”, la mole di nuovi concetti da imparare potrebbe sembrarvi un po’ opprimente al primo impatto. Posso solo consigliarvi di non stressarvi troppo. Cercate di imparare passo dopo passo, di divertirvi e di progredire secondo i vostri ritmi. 
Nel giro di pochi giorni, dovreste già conoscere il programma abbastanza da poter comporre i vostri primi pezzi. 

Questo manuale è stato principalmente scritto come una guida per chi parte da zero, ma vuol essere anche un completo manuale di riferimento. Tuttavia, ci sono molte informazioni che non riuscirebbero a stare in un manuale come questo. Vi consiglio caldamente di guardare il Wiki gestito dagli utenti al link \url{http://wiki.littlesounddj.com} - contiene materiali come tutorial, consigli, trucchi e progetti DIY. Inoltre, anche il gruppo Facebook \url{https://www.facebook.com/groups/LittleSoundDJ/} può essere utile per mantenersi in contatto con altri utilizzatori di LSDJ.
Se avete domande, idee o bug da segnalare, potete inviarmi una mail a \href{mailto:info@littlesounddj.com}{info@littlesounddj.com}.

Felice tracking!

\textit{/Johan}

\section{Avviso Importante}

Spegnere il Game Boy mentre state suonando può farvi perdere la vostra canzone, quindi vi consiglio di evitarlo.
Inoltre, è meglio non usare il programma quando le batterie sono quasi scariche, per evitare che il Game Boy si spenga da solo.
Un basso livello delle batterie è indicato dall'affievolirsi della luce rossa sul vostro Game Boy, o dall'affievolirsi della luminosità dello schermo.

\section{Il suono del Game Boy}
Il chip sonoro del Game Boy ha quattro canali, ognuno dei quali ha una risoluzione di 4-bit:

\begin{description}
\item[Pulse Channel 1] Onda quadra con funzioni di inviluppo (envelope) e sweep (DA CHIARIRE)..
\item[Pulse Channel 2] Onda quadra con funzioni di inviuppo.
\item[Wave Channel] Sintetizzatore software, playback di campioni e sintesi vocale.
\item[Noise Channel] Rumore (noise) con funzioni di inviluppo e forma (shape).
\end{description}


\section{Pulsanti}
In questo manuale, la pressione su di un tasto è segnalata in questo modo:
\begin{description}
\item[\textsc{a}] tasto \textsc{a} 
\item[\textsc{b}] tasto \textsc{b} 
\item[\textsc{start}] tasto start
\item[\textsc{select}] tasto select
\item[\textsc{cursor}] premere una qualsiasi direzione sulla croce direzionale (frecce) 
\item[\textsc{left}] freccia direzionale sinistra
\item[\textsc{right}] freccia direzionale destra
\item[\textsc{up}] freccia direzionale in alto
\item[\textsc{down}] freccia direzionale in basso
\item[\textsc{left/right}] freccia direzionale sinistra o destra
\item[\textsc{up/down}] freccia direzionale in alto o in basso
\item[\textsc{select+a}] premere \textsc{a} tenendo premuto \textsc{select}
\item[\textsc{select+(b,b)}] premere \textsc{b} due volte, tenendo premuto \textsc{select}
\end{description}

\section{Navigare Nel Programma}
Dopo aver avviato LSDj, vedrai una schermata come quella in Figura~\ref{fig:song}.

\begin{figure}[hbtp]
\centering
\fbox{ \includegraphics{song} }
\caption{Song Screen}
\label{fig:song}
\end{figure}

Il titolo \textsc{song} in alto a sinistra, indica che siete nel Song Screen (Schermo Canzone), ovvero la finestra dove organizzare i vostri pezzi. Ognuna delle quattro colonne coi trattini rappresenta un diverso canale del chip sonoro del Game Boy. Ci sono due canali pulse wave (onde quadre - PU1 e PU2), un canale per le forme d’onda create dall’utente (che utilizza kit di suoni precampionati, o forme d’onde gestite da un sintetizzatore software), e un canale per il noise.
Puoi spostarti da un canale all’altro utilizzando la croce direzionale.


\begin{figure}[hbtp]
\centering
\fbox{ \includegraphics[width=2.5cm]{map} }
\caption{Screen Map}
\label{fig:map}
\end{figure}

Little Sound Dj usa nove schermate differenti, organizzate nella mappa visibile sullo schermo in basso a destra (figure~\ref{fig:map}).  Potete navigare fra i diversi schermi premendo contemporaneamente \textsc{select} e la croce direzionale. 

Le schermate più utili sono posizionate nella riga centrale della mappa, ordinate gerarchicamente per livello di dettaglio.
Gli schermi song (canzone), chain (catena) e phrase (frase) sono utilizzati per il sequencing e funzionano secondo una struttura ad albero. Ogni canzone (song) è formata da più catene (chain), ogni chain contiene più frasi (phrase), e ogni frase contiene delle note. A seguire, abbiamo la schermata Instrument (strumento) e la schermata Table (tabella), che servono a creare i suoni.

\section{Costruire I Vostri Primi Suoni}


Navigate fino allo schermo song e muovete il cursore sulla colonna \textsc{pu1}. Ora premete il bottone \textsc{a} una volta, per creare una nuova Chain “00”.
Potete ora modificare la chain premendo \textsc{select+right} per andare sulla schermata chain.
A questo punto, premete \textsc{a} per inserire una nuova phrase, poi premete \textsc{select+right} per andare sulla schermata phrase.

\begin{figure}[hbtp]
\centering
\fbox{\includegraphics{phrase}}
\caption{Phrase Screen}
\label{fig:phrase1}
\end{figure}

Nella schermata Phrase, potete inserire le note da riprodurre. Spostate il cursore sulla colonna note e premete \textsc{a} per inserire una nota. Apparirà il testo C-2: C è la nota, mentre 2 è l’ottava. Premete \textsc{start} per riprodurre la frase. Come potete vedere, le note vengono eseguite dall'alto verso il basso. Potete cambiare la nota tenendo premuto \textsc{a} e muovendo la croce direzionale. \textsc{a+left/right} + cambia la nota, mentre \textsc{a+up/down} cambia l’ottava.

Adesso, provate a muovere il cursore e a inserire note sugli altri step. Per cancellare una nota, premete \textsc{a} mentre state schiacciando \textsc{b}. Quando avete finito di ascoltare il vostro lavoro, premete nuovamente \textsc{start} per fermare la phrase. 

Il semplice suono del canale pulse può suonare monotono dopo un po'. Per questa ragione, spostatiamoci sulla schermata instrument premendo \textsc{select+right}.

\begin{figure}[hbtp]
\centering
\fbox{\includegraphics{instr-pulse}}
\caption{Instrument Screen}
\label{fig:instr}
\end{figure}

Nella schermata instrument, possiamo rendere il suono un po' più interessante. Provate a modificare i campi \textsc{adsr} e \textsc{wave}, selezionandoli col cursore e premendo \textsc{a+left/right}. Ad esempio, impostare il nostro ADSR su \texttt{A3} dovrebbe  rendere il nostro suono un po' più rimbalzante. Premete nuovamente \textsc{start}, per ascoltare le modifiche che state apportando.

Il campo \textsc{type} definisce il tipo di strumento. Ogni canale prevede l'uso di un particolare tipo di strumento -- Gli strumenti \textsc{pulse} dovrebbero essere utilizzati solo sui canali pulse, gli strumenti \textsc{wave} e \textsc{kit} sul canale wave e gli strumenti \textsc{noise} sul canale noise.

Proviamo adesso i kit di campioni. Per prima cosa, dobbiamo spostarci sul canale wave. Andiamo indietro alla schermata song, muoviamo il cursore sul canale wave e creiamo una nuova chain e una nuova phrase premendo \textsc{a}. Inserite una nota premendo \textsc{a}, poi usate \textsc{select+right} per andare sull'editor dell'instrument. Premendo \textsc{a+right} sulla sezione field, andiate a selezionare \textsc{kit} come tipo di strumento, poi tornate alla schermata phrase. Adesso, dovreste essere in grado di inserire i suoni di batteria nello stesso modo in cui avete già inserito le note.

Per creare nuove chain e nuove phrases, selezionate uno step vuoto nella schermata song o nella schermata chain. Poi, premete \textsc{a} due volte.

\section{Risoluzione Dei Problemi Iniziale}

La vostra cartuccia non parte, va in crash, o si comporta in modo strano? Ecco alcune soluzioni che puoi provare.

\begin{itemize}
\item Pulite i pin della cartuccia usando un batuffolo di cotone e dell'alcol.
\item Re-inserite la cartuccia un paio di volte, per rimuovere eventuale ossido.
\item Assicuratevi che la cartuccia sia saldamente inserita. A volte può essere utile mettere un pezzo di nastro sulla cartuccia, per inspessirla e darle una forma più aderente.
\item Cambiate le batterie, inserendo delle batterie nuove.
\item Eseguite un reset completo della memoria della cartuccia, premendo \textsc{select+a+b} sul pulsante \textsc{load/save file} nella schermata progetto.
\item Alcune cartucce per Game Boy Advance/Nintendo DS non funzionano con Little Sound Dj. Se avete problemi con cartucce di questo tipo, provate una delle build di Little Sound Dj nominate ``Goomba''.
\item Chiedete aiuto sulla Wiki di Little Sound Dj (\url{http://wiki.littlesounddj.com}) o sul gruppo Facebook.
\end{itemize}

\section{Il Sistema Numerico Esadecimale}

Prima di passare al capitolo successivo, è giunto il momento di introdurre il sistema numerico esadecimale, che Little Sound Dj usa per rappresentare i valori numerici. 

Il sistema numerico esadecimale funziona esattamente come il sistema decimale
tradizionale. L'unica differenza è che invece di essere in base 10 è in base 16. Ciò significa
che è formato da 16 simboli unici: le cifre da 0 a 9, seguite dalle lettere dalla A alla F. Per
maggiore chiarezza, questo manuale indicherà i valori esadecimali con il simbolo del dollaro: \$. Come
esempio, riportiamo una tabella di numeri, scritti sia con cifre decimali che esadecimali\ldots

\begin{figure}[hbtp]
\centering

\begin{tabular}{r|r|r|r|r|r|r|r|r|r|r}
 Decimal & 1 & 2 & 3 & 4 & 5 & 6 & 7 & 8 & 9 & 10 \\
\hline
 Hexadecimal & \$1 & \$2 & \$3 & \$4 & \$5 & \$6 & \$7 & \$8 & \$9 & \$A \\
\end{tabular}

\begin{tabular}{r|r|r|r|r|r|r|r|r|r|r}
 Decimal & 11 & 12 & 13 & 14 & 15 & 16 & 17 & 18 & 19 & 20 \\
\hline
 Hexadecimal & \$B & \$C & \$D & \$E & \$F & \$10 & \$11 & \$12 & \$13 & \$14  \\
\end{tabular}

\end{figure}

Potete osservare come i valori esadecimali siano veramente uguali a quelli decimali; cambia solo la loro rappresentazione. La ragione per cui viene utilizzato il sistema esadecimale è che permette di risparmiare spazio sullo schermo; con numeri esadecimali è possibile rappresentare ogni valore byte utilizzando non più di due cifre (L'intervallo di valori utilizzato
da LSDj va da 0 a 255 – quindi da \$0 a \$FF). 

Rappresentare i numeri negativi con solo due cifre può essere un problema. In Little Sound Dj, i numeri sono “circolari”. Questo significa che sottraendo 1 dal più piccolo
numero possibile (\$0), il risultato diventerà il più alto valore possibile (\$FF). Di conseguenza, \$FF può voler dire sia -1 sia 255, a seconda della situazione. 

Non preoccupatevi troppo, se non riuscite a capire immediatamente questa spiegazione. Diventerà tutto molto chiaro con l'utilizzo del programma. 
