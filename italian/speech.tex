\chapter{Programmare la voce sintetica}
\label{speech-chapter}

\section{Introduzione}

Little Sound Dj contiene 59 suoni vocali, chiamati allofoni. Combinando questi suoni è possibile creare ogni possibile parola o frase in inglese.

Lo strumento speech (parlato) è fisso allo strumento numero \$40 e può essere usato solo nel canale wave. Contiene 42 parole, mappate sulle note da C-2 ad F-5.

\begin{figure}[htpb]
	\begin{center}
	\fbox{\includegraphics{speech}}
	\end{center}
	\caption{Schermo dello strumento speech}
	\label{fig:speech}
\end{figure}
\begin{figure}[htpb]
	\begin{center}
	\fbox{\includegraphics{word}}
	\end{center}
	\caption{Parola d'esempio}
	\label{fig:word}
\end{figure}

Per modificare una parola, premere \textsc{select+right} per passare allo schermo word. La colonna di sinistra contiene gli allofoni da riprodurre, la colonna di destra la loro durata. La parola nella figura~\ref{fig:word} è programmata per dire ``Little Sound Dj''.

Per rendere le parole facili da ricordare, è possibile rinominarle premendo A su di esse nello schermo dello strumento speech.

\section{Allofoni}

\begin{itemize}
\item Quando si selezionano gli allofoni, è importante tenere conto di come suonano, non di come si scrivono letterale della parola da rappresentare.
\item Un suono può essere differente a seconda della sua posizione all'interno di una parola. Ad esempio, la K in ``coop'' suonerà diversa dalle K in ``keep'' e ``speak''.
\end{itemize}

Gli allofoni marchiati con * sono riprodotti in loop, all'infinito.

Gli allofoni di LSDj sono comunque tarati su suoni propri della lingua inglese e potrebbero non avere una buona resa in italiano.

\subsection{Vocali brevi}

\begin{description}
\item[*IH] sitting, stranded
\item[*EH] extent, gentlemen
\item[*AE] extract, acting
\item[*UH] cookie, full
\item[*AD] talking, song
\item[*AX] lapel, instruct
\end{description}

\subsection{Vocali lunghe}

\begin{description}
\item[IY] treat, people, penny
\item[EY] great, statement, tray
\item[AY] kite, sky, mighty
\item[OI] noise, toy, voice
\item[UW1] dopo gruppi consonantici con YY: computer
\item[UW2] nelle parole monosillabiche: two, food
\item[OW] zone, close, snow
\item[AW] sound, mouse, down
\item[EL] little, angle, gentlemen
\end{description}

\subsection{Vocali retroflesse}

\begin{description}
\item[ER1] letter, furniture, interrupt
\item[ER2] monosillabiche: bird, fern, burn
\item[OR] fortune, adorn, store
\item[AR] farm, alarm, garment
\item[YR] hear, earring, irresponsible
\item[XR] hair, declare, stare
\end{description}

\subsection{Sonanti}
\begin{description}
\item[WW] we, warrant, linguist
\item[RR1] posizione iniziale: read, write, x-ray
\item[RR2] gruppo iniziale: brown, crane, grease
\item[LL] like, hello, steel
\item[YY1] nei gruppi consonantici: cute, beauty, computer
\item[YY2] in posizione iniziale: yes, yarn, yo-yo
\end{description}

\subsection{Fricative sonore}
\begin{description}
\item[VV] vest, prove, even
\item[DH1] iniziale della parola: this, then, they
\item[DH2] finale della parola e fra vocali: bathe, bathing
\item[ZZ] zoo, phase
\item[ZH] beige, pleasure
\end{description}

\subsection{Fricative sorde}
\begin{description}
\item[*FF] fire, fox
\item[*TH] this, they
\item[*SS] sit, smile
\item[SH] shirt, leash, nation
\item[HH1] prima delle vocali anteriori: YR, IY, IH, EY, EH, XR, AE
\item[HH2] prima delle vocali posteriori: UW, UH, OW, OY, AO, OR, AR
\item[WH] white, whim, twenty
\end{description}

\subsection{Consonanti occlusive sonore}
\begin{description}
\item[BB1] in posizione finale: rib; fra vocali: fibber, nei gruppi consonantici: bleed, brown
\item[BB2] posizione iniziale prima di una vocale: beast
\item[DD1] posizione finale: played, end
\item[DD2] posizione iniziale: down; clusters: drain
\item[GG1] prima delle vocali anteriori chiuse non arrotondate: YR, IY, IH, EY, EH, XR
\item[GG2] prima delle vocali posteriori chiuse arrotondate before: UW, UH, OW, OY, AX; e nei gruppi consonantici: green, glue
\item[GG3] prima delle vocali aperte: AE, AW, AY, AR, AA, AO, OR, ER; nei gruppi di consonanti a metà parola: anger; in posizione finale: peg
\end{description}


\subsection{Consonanti occlusive sorde}
\begin{description}
\item[PP] pleasure, ample, trip
\item[TT1] gruppi di consonanti prima di SS: tests, its
\item[TT2] tutte le altre posizioni: test, street
\item[KK1] prima delle vocali anteriori: YR, IY, IH, EY, EH, XR, AY, AE, ER, AX; gruppi di consonanti a inizio parola: cute, clown, scream
\item[KK2] in posizione finale: speak; nei gruppi di consonanti a fine parola: task
\item[KK3] prima delle vocali posteriori: UW, UH, OW, OY, OR, AR, AO; gruppi di consonanti a inizio parola: crane, quick, clown, scream
\end{description}

\subsection{Consonanti affricate}
\begin{description}
\item[CH] church, feature
\item[JH] judge, injure
\end{description}

\subsection{Nasali}
\begin{description}
\item[MM] milk, alarm, example
\item[NN1] prima delle vocali anteriori e centrali: YR, IY, IH, EY, EH, XR, AE, ER, AX, AW, AY, UW; nei gruppi di consonanti a fine parola: earn
\item[NN2] prima delle vocali posteriori: UH, OW, OY, OR, AR, AA
\end{description}



